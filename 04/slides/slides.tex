\documentclass[aspectratio=169]{beamer}

% SETUP =====================================
\usepackage[T1,T2A]{fontenc}
\usepackage[utf8]{inputenc}
\usepackage[russian]{babel}
\usepackage{listings}
\usepackage{array}
\usepackage{amssymb}
\usepackage{pifont}
\usepackage{minted}
\usepackage{../../beamerthemeslidesgeneric}
% SETUP =====================================

\title{Scala Implicits}
\author{Mikhail Mutcianko, Alexey Shcherbakov}
\institute{СПБгУ, СП}
\date{4 марта 2021}

\begin{document}

\frame{\titlepage}

\section{Scala function parameters}

\begin{frame}[fragile]{Parameter lists}
  Scala allows function definitions with multiple parameter lists \\
\begin{minted}{scala}
def multiply(a: Int, b: Int): Int|\pause|
def multiply(a: Int)(b: Int): Int|\pause|

multiply(2,3)
multiply(2)(3)
\end{minted}
\end{frame}

\begin{frame}[fragile]{Omitting parenthesis}
If parameters to a function are not required at call-site, the parenthesis can be omited \\
\begin{minted}{scala}
def foo(x: Int = 123)()

foo() // OK(with a warning)
\end{minted}
\end{frame}

\section{Implicit parameters}

\begin{frame}{Implicit parameters}
  \begin{block}{}
    A method can have an implicit parameter list, marked by the \alert{implicit} keyword at the
    start of the parameter list.\vspace{1em}\\If the parameters in that parameter list are not
    passed as usual, Scala compiler will look if it can get an implicit value of the correct type,
    and if it can, pass it automatically. 
  \end{block}
\end{frame}

\begin{frame}[fragile]{Implicit parameters}{Example}
\begin{onlyenv}<1>
  \begin{minted}{scala}
  ???

  def findInt(implicit int: Int): Int = int

  findInt // res0 = 123
  \end{minted} 
\end{onlyenv}
\begin{onlyenv}<2>
  \hig{1}
  \begin{minted}{scala}
  implicit val i: Int = 123

  def findInt(implicit int: Int): Int = int

  findInt // res0 = 123
  \end{minted}
\end{onlyenv}
\end{frame}

\begin{frame}{Trivia}
  \begin{itemize}
    \item all parameters in a clause will be implicit
    \item only one implicit clause allowed per function
    \item implicit clause must be the last one
    \item all parameters must be found to omit the clause
    \item implicit parameters can be passed explicitly
  \end{itemize}
\end{frame}

\begin{frame}{Application rules}
  Say, a function takes an implicit parameter of type \texttt{T}. \\
  The compiler will search an implicit definition that:
  \bigskip
  \begin{itemize}
    \item is marked \texttt{implicit}
    \item has a type compatible with \texttt{T}
    \item is visible at the point of the function call
      or is defined in a companion object associated with \texttt{T}
  \end{itemize}
  \bigskip
  If there is a single (most specific) definition, it will be taken as actual argument for the
  implicit parameter. Otherwise it's an error.
\end{frame}

\begin{frame}[fragile]{Passing implicit parameters}
\alert{implicit} parameters keep their \textit{implicitness} inside a function definition
\bigskip
\begin{minted}{scala}
def foo(i: Int)(implicit ctx: Context)    = bar(i.toString) // ctx is passed to bar
def bar(s: String)(implicit ctx: Context) = println(s)

implicit def c: Context = ???
foo(42)
\end{minted}
\end{frame}

\section{Implicit definitions}

\begin{frame}[fragile]{Implicit definitions}
Implicit parameter is substisuted as a value. Value definitions in Scala can be marked as 
\alert{implicit} \vspace{1em}
\begin{onlyenv}<1>
  \begin{minted}{scala}
  trait Context
  
  def foo(i: Int)(implicit ctx: Context)    = bar(i.toString)
  def bar(s: String)(implicit ctx: Context) = println(s)
  
  implicit val c: Context = new Context {}
  foo(42)
  \end{minted}
\end{onlyenv}
\begin{onlyenv}<2>
  \hig{6}
 \begin{minted}{scala}
  trait Context
  
  def foo(i: Int)(implicit ctx: Context)    = bar(i.toString)
  def bar(s: String)(implicit ctx: Context) = println(s)
  
  implicit def c: Context = new Context {}
  foo(42)
  \end{minted} 
\end{onlyenv}
\begin{onlyenv}<3>
  \hig{6}
 \begin{minted}{scala}
  trait Context
  
  def foo(i: Int)(implicit ctx: Context)    = bar(i.toString)
  def bar(s: String)(implicit ctx: Context) = println(s)
  
  implicit object c extends Context {}
  foo(42)
  \end{minted} 
\end{onlyenv}
\end{frame}

\begin{frame}[fragile]{Definition restriction}
\texttt{implicit} definitions cannot be top-level
\bigskip
\begin{minted}{scala}
//file Foo.scala
package com.foo.bar

implicit object Foo { 
 // ^--- error: `implicit' modifier cannot be used for top-level objects
}
\end{minted}
\end{frame}

\section{Implicit Conversions}

\begin{frame}[fragile]{Implicit conversion}
  \begin{block}{}
    When a the provided type doesn't match the expected type, the Scala compiler looks for any
    method in scope marked implicit that takes the provided type as parameter and returns the
    expected type as a result. If found, it inserts the call to the method in between.
  \end{block}
  \bigskip
  %\pause
  \begin{onlyenv}<1>
   \begin{minted}{scala}
    implicit def int2String(i: Int): String = i.toString()

    123.startsWith("12")
   \end{minted} 
  \end{onlyenv}
  \begin{onlyenv}<2>
    \hig{3}
     \begin{minted}{scala}
      implicit def int2String(i: Int): String = i.toString()

      int2string(123).startsWith("12")
     \end{minted} 
    \end{onlyenv}
\end{frame}

\begin{frame}[fragile]{Implicit conversions}{Limitations}
Implicit conversions \textbf{cannot} be nested at \textit{application-site}\\ but \textbf{can}
be nested during \textit{resolve}
\medskip
\begin{onlyenv}<1>
\begin{minted}{scala}
case class A(i: Int)
case class B(i: Int)
case class C(i: Int)

implicit def aToB(a: A): B = B(a.i)
implicit def bToC(b: B): C = C(b.i)

val a = A(1)
val c: C = a // error: type mismatch, expected C got A
\end{minted} 
\end{onlyenv}
\begin{onlyenv}<2>
\hig{6}
\begin{minted}{scala}
case class A(i: Int)
case class B(i: Int)
case class C(i: Int)

implicit def aToB(a: A): B = B(a.i)
implicit def bToC[T](t: T)(implicit tToB: T => B): C = C(t.i)

val a = A(1)
val c: C = a // OK
\end{minted}
\end{onlyenv}
\end{frame}

\begin{frame}[fragile]{Implicits class}
\begin{block}{}
   An \alert{implicit class}  is a class marked with the \texttt{implicit} keyword. This keyword
   makes the class’s primary constructor available for implicit conversions when the class is in
   scope
\end{block}
\bigskip
\begin{onlyenv}<1>
  \begin{minted}{scala}
implicit class RichString(val str: String) {
  def toUrl: URL = new URL(str)
}

"https://foo.com".toUrl
\end{minted}
\end{onlyenv}
\begin{onlyenv}<2>
  \hig{1}
  \begin{minted}{scala}
implicit class RichString(val str: String) extends AnyVal {
  def toUrl: URL = new URL(str)
}

"https://foo.com".toUrl
\end{minted}
\end{onlyenv}
\begin{onlyenv}<3>
  \hig{5}
  \begin{minted}{scala}
implicit class RichString(val str: String) extends AnyVal {
  def toUrl: URL = new URL(str)
}

new RichString("https://foo.com").toUrl
\end{minted}
\end{onlyenv}
\end{frame}

\begin{frame}[fragile]{Implicit class}{Limitations}
1. They must be defined inside of another trait/class/object
\bigskip
\begin{minted}{scala}
object Helpers {
   implicit class RichInt(x: Int) // OK!
}

implicit class RichDouble(x: Double) // BAD!
\end{minted}
\end{frame}


\begin{frame}[fragile]{Implicit class}{Limitations}
2. They may only take one non-implicit argument in their constructor
\bigskip
\begin{minted}{scala}
implicit class RichDate(date: java.util.Date) // OK!
implicit class Indexer[T](collection: Seq[T], index: Int) // BAD!
implicit class Indexer[T](collection: Seq[T])(implicit index: Index) // OK!
\end{minted}
\end{frame}


\begin{frame}[fragile]{Implicit class}{Limitations}
3. There may not be any method, member or object in scope with the same name as the implicit
class\\[1em]
\alert{NB!} \textit{This means an implicit class cannot be a case class}
\bigskip
\begin{minted}{scala}
object Bar
implicit class Bar(x: Int) // BAD!

val x = 5
implicit class x(y: Int) // BAD!

implicit case class Baz(x: Int) // BAD!
\end{minted}
\end{frame}

\section{Implicit Search}

\begin{frame}[fragile]{1 - Local scope}
\bigskip
\begin{minted}{scala}
trait Show[A] { def show(a: A): String }

implicit val showInt: Show[Int] = new Show[Int] {
  def show(a: Int): String = a.toString
}

def stringify[A](a: A)(implicit s: Show[A]): String = s.show(a)
stringify(1)
\end{minted}
\end{frame}

\begin{frame}[fragile]{2 - Explicit imports}
\bigskip
\begin{minted}{scala}
trait Show[A] { def show(a: A): String }

def stringify[A](a: A)(implicit s: Show[A]): String = s.show(a)

import Show.instances.intShow
stringify(1)
\end{minted}
\end{frame}

\begin{frame}[fragile]{3 - Wildcard imports}
\bigskip
\begin{minted}{scala}
trait Show[A] { def show(a: A): String }

def stringify[A](a: A)(implicit s: Show[A]): String = s.show(a)

import Show.instances._
stringify(1)
\end{minted}
\end{frame}

\begin{frame}[fragile]{4 - Companion object of type}{or Implicit scope of an argument’s type}
\bigskip
\begin{minted}{scala}
trait Show[A] { def show(a: A): String }
case class Foo(a: Int)
object Foo {
  implicit val fooShow: Show[Foo] = {f: Foo => f.toString}
}

def stringify[A](a: A)(implicit s: Show[A]): String = s.show(a)
stringify(Foo(1))
\end{minted}
\end{frame}

\begin{frame}[fragile]{5 - Implicit scope of type arguments}{or Companion object of a Type Class}
\begin{minted}{scala}
  case class Foo(i: Int)
  object Foo {
    implicit def showFoo: Show[Foo] = {f: Foo => f.i.toString}
  }
  trait Show[A] {def show(a: A): String}
  object Show {
    implicit def showOpt[A](implicit s: Show[A]): Show[Option[A]] = ...
  }
  def stringify[A](a: A)(implicit s: Show[A]): String = s.show(a)
  stringify(Option(Foo(1)))
\end{minted}
\end{frame}

\section{Implicit's Application Patterns Intro}

\begin{frame}[fragile]{Library syntax}{Context bounds}
Context bounds are basically syntactic sugar for less verbose typeclass applications.\\
The following lines are identical:
\bigskip
\begin{minted}{scala}
def g[A : B](a: A) = implicitly[B[A]].h(a)
def g[A](a: A)(implicit ev: B[A]) = ev.h(a)
\end{minted}
\end{frame}

\begin{frame}[fragile]{Library syntax}
Split typeclasses and syntax enrichments via implicit conversions and context bounds
\begin{minted}{scala}
object Show {
  object syntax {
    implicit class ShowSyntax[A](val a: A) extends AnyVal {
      def stringify(implicit s: Show[A]): String = s.show(a)
} } }

import Show.syntax._
def stringify[A: Show](a: A): String = a.stringify

stringify(1)
1.stringify
\end{minted}
\end{frame}

\begin{frame}{Context pattern}
  \begin{block}{}
    The most basic use of implicits is the Implicit Context pattern: using them to pass in some
    "context" object into all your methods.\\This is something you could pass in manually, but is
    common enough that simply "not having to pass it everywhere" is itself a valuable goal.
  \end{block}
\end{frame}

\begin{frame}[fragile]{Context pattern}{Example}
\begin{minted}{scala}
class Context
def foo(i: Int)(implicit context: Context): String = ???
def bar(s: String)(implicit context: Context): String = ???
def baz(i: Int)(implicit context: Context): String = bar(foo(i))

implicit def context = new Context
baz(1)
\end{minted}
\end{frame}


\begin{frame}[fragile]{Implicit derivation}
One neat thing about using Type-class Implicits is that you can perform "deep" checks.
\hig{7}
\begin{minted}{scala}
  case class Foo(i: Int)
  trait Show[A] {def show(a: A): String}
  object Show {
    implicit def showFoo: Show[Foo] =
      {f: Foo => f.i.toString}
    implicit def showSeq[A: Show]: Show[Seq[A]] =
      (a: Seq[A]) => a.map(x => implicitly[Show[A]].show(x)).mkString
  }
  def stringify[A](a: A)(implicit s: Show[A]): String = s.show(a)
  stringify(Seq(Seq(Foo(1))))
\end{minted}
\end{frame}


%\begin{thebibliography}{9} \bibitem{spec-lin}
%\url{https://www.scala-lang.org/files/archive/spec/2.11/05-classes-and-objects.html#class-linearization}
%\end{thebibliography}

\end{document}

